\chapter*{About this project}
\paragraph{Abstract}
A brief description of what the project is, in about two-hundred and fifty words.

\paragraph{Authors}
Explain here who the authors are.



\chapter{Introduction}
\bigbreak
It’s a booking web application build using MEAN stack (MongoDB, Express, Angular, Node).The application is made of two websites bookAroom-user and bookAroom-admin.
BookAroom-user is using Auth0 authentication system which allows for convenient log in with gmail fb as well as sign up form with password recovery. Once user is signed in he/she can book search and sort workbenches.
BookAroom-admin is using json web-tokens identification system. It lets administrative user to add, edit and deletes existing and new workbenches.

\chapter{Context}
\begin{itemize}
\item Provide a context for your project.
\item Set out the objectives of the project
\item Briefly list each chapter / section and provide a 1-2 line description of what each section contains.
\item List the resource URL (GitHub address) for the project and provide a brief list of the main elements at the URL.
\end{itemize}

\section{Filler}


\section{Filler}


\chapter{Methodology}
About one to two Page
Describe the way you went about your project:
\begin{itemize}
\item Agile / incremental and iterative approach to development. Planning, meetings.
\item What about validation and testing? Junit or some other framework.
\item If team based, did you use GitHub during the development process.
\item Selection criteria for algorithms, languages, platforms and technolo-gies.
\end{itemize}



\chapter{Technology Review}
\section{Development Environment}
About seven to ten pages.
\subsection{Debian}
\bigbreak
Debian is an open source operating System developed. It’s one of many distributions of Linux but is one of the earliest distributions and was first announced in 1993 by Ian Murdock. It requires small hardware resources to run and is easy to set up and install it has graphical user interface version as well as command line version which father. When distribution is released initially it’s marked as unstable, father sufficient amount testing is done it become as stable and usually is adopted by more people than because it has less bugs and is more reliable. Debian has access to over 50 000 to packages located on the internet and any package can be installed with one command line which makes ideal work environment for web development because web development relies on many different components put together and if not done right issues arise with dependencies etc. With commands like sudo apt-get you can install any package you want and well documented and big community behind it.Tools like npm Node packet manger and grunt work very well in this environment.

\subsection{Cli and Gnu Nano}
\bigbreak
Command Line Interface (CLI) doesn’t come with Debian it has to be installed. Moust of Operating Systems come with many tools installed which makes them very big and a bit slow Operating System like Debian takes different approach and start with just essential tools and then you install only tools that you need. CLI is needed to install some packages and execute some of the commands in Debian when working web environment.GNU Nano is a text editor Unix like operating system which makes it ideal for Debian and all Linux distributions. It allows for convenient bash scripts to be set up while in command prompt or terminal as now in Debian. As well it allows for convenient GitHub commits you can enter commit message and press chines hat and O to write out and commit changes to github.


\subsection{Npm/Bower Grunt}
\bigbreak
These two tools are absolutely essential to web development environment and you are must like wouldn’t be able to develop without them in today’s world. Npm is a default manager for JavaScript runtime environment in Node.js.it manages all package dependencies of an application as well as installing them. Grunt makes web development environment more efficient by eliminating repetitive tasks such as unit testing as well as shortening your run scripts such grunt serve is very popular for running your application as that is very repetitive command and is executed a lot.


\bigbreak

Bower its package management system for web applications and uses node.js and npm.
In this project bower installation method was used preferably to npm. When installing firebase to the application throw npm using commands like.

\begin{itemize}
	\item npm install firebase –save
\end{itemize}
Reference needs to be added to index.html page which downloads firebase module over the net.


\begin{minted}{javascript}
src="https://cdn.firebase.com/js/client/2.2.4/firebase.js"
\end{minted}




\subsection{Github}
\bigbreak
Github was founded in 2008 its web based git repository it lets users have public repositories which are free and you can have any amount it also has an option for private repositories for which you have to pay and the more private repositories you have more you have to pay, but I has student bundles which provide 100 dollars’ worth of development tools which is great for beginner developers such as myself. It provides graphical user interface for Windows and Mac to manage maintain and update projects but it does cause some errors failing to commit or failing to push or merge when that happens user is suggested to use git to fix the errors. For that reason git is way more stable and reliable which is command based tool it has five major commands hat are used constantly.It has played major role in this project and made project way more manageable and sustainable as well as less error prone regular commits made sure of small incremental progress throw out life of the project.

\begin{itemize}
	
\item git clone path/to/repo \\
\\This command allows to clone any project from your github account or any publicly available project for that matter.

\item git status \\
\\Checks the status of project once it has been cloned it shows added deleted and modifies files as well as indicates which files are tracked.

\item git add . \\ 
\\Add folders or files to be tracked by github so later they can be committed and pushed on to github account.

\item git commit –a \\
\\Makes local commit of all the changes done in folder added by previous command git add .

\item git push origin master \\ 
\\ Pushes commit (git commit -a) to users github account on his/her account

\end{itemize}


\subsection{Sublime, Live Reload and Chrome}
\bigbreak
Sublime is code editor that supports cross platform functionality. It supports many different programming languages as well as mark-up languages and functionality can be extended with huge amount of plug in packages available. Currently Sublime is on version 3 it has been released in 2013.
There are other editors out there such as Notepad++, Brackets, Vim Atom but Sublime proved to be most efficient in web development environment.Some of its best features;

\begin{itemize}
	
	\item auto safe 
	\item autocompleting 
	\item multi-select-editing 
	\item spell check 
	\item snippets  

\end{itemize}





\section{Mean Stack}
\subsection{Mongodb/Firebase}
\bigbreak
MongoDB comes as part of popular stack MEAN it’s a cross-platform document-oriented database and is classified as NoSql database. Its free and open source under a combination of the GNU Affero General Public License and the Apache License. Being object oriented data base it is shameless and needs no scheme as its older counter parts SQL based data bases. For that matter there is mongoose was developed which is an ORM for Mongo and is written in node.js and allows to give mongodb a scheme to make it more comparable with older system which all use SQL based data bases. It has grown in popularity and now is fourth most popular database management system. Its stores everything as an object in JSON like format and files object are indexed which allows instant retrieval of an object. 
\\
\bigbreak
Firebase was founded in 2011 and is object oriented database allowing for fast retrieval of objects. It allows to store and sync data across many different platforms one of its best feature is three way data binding between Firebase and your applications view and controller and stores data in json like format which is well understood format by developers across the world. Company has been acquired by google in 2014. Firebase is built in to Auth0 authentication framework which was one of the reasons why it was chosen for this project because Auth0 framework was used for authentication of this application Auth0 framework will be explained in more detail in Components section. Firebase generates authentication key which needs to be supplied to Auth0 in order to let firebase requests throw authentication system shown below.
\begin{minted}{javascript}

var ref = new Firebase("https://bookaroomfirebase.firebaseio.com/");
ref.auth("AUTH_TOKEN", function(error, result){
if (error){
console.log("Authentication Failed!", error);
}else{
console.log("Authenticated successfully with payload:", result.auth);
console.log("Auth expires at:", new Date(result.expires * 1000));
}
});

\end{minted}


\subsection{Express}
\bigbreak
Is part of the MEAN stack bundle and it handles server side. Express is owned by IBM as of 2015 ans in 2016 IBM announced that it will put Exprrss.js under the stewardship of the Node.js foundation incubator.
\bigbreak
Express.js is a node.js web application (node.js is covered in other section) and uses minimalistic approach all server side script can be done in few lines of code this peace of code was used to run bookAroom-admin website shown below.

\begin{minted}{javascript}

var express = require('express');
var app = express();
var port = process.env.PORT || 8080;

app.use(express.static(__dirname + '/dist/'));

app.listen(port, function() {
console.log('Our app is running on port: ' + port);
});

\end{minted}
\bigbreak
And everything else is available as a plug in you simply require all the modules that you need. The modules below where used in bookAroom-user application to authenticate user.

\begin{minted}{javascript}

var http = require('http')
var cors = require('cors');
var jwt = require('express-jwt');
var dotenv = require('dotenv');

\end{minted}


\subsection{Angular}
\bigbreak
Web applications are browser-based applications running in a browser
using HTML5. WebHooks allow developers to access the hardware on a
phone, this was unavailable before HTML5 [16], also it allows other features
such as web storage, indexed database APIs, file APIs, web SQL Databases and
Offline Web GeoLocations [16]. This makes web applications more mobile
friendly and compatible and allows them to use the full range of phone
features. They do not require installation or any upgrades as it contains a one
to many relationship (one server, many clients) so any updates are done on
the server side and all clients get updated, but the network is required at all
times in order to access the application. It lacks the native look and feel of
target platforms such as Android, iOS or Windows Phone, altough there are
many tools out there trying to solve the problem by simulating a native look
such as Xui, JQyeryMobile, Sencha Touch, JQTouch and WebApp.net. Some
6
frameworks developing web applications include AngularJS, Ruby on Rails,
Django and Drupal. AngularJS is explained in more detail down below.
\bigbreak
It was developed by Misko Hevery in 2009 at Brath Tech LLC. It is now an
open source framework mainly used for developing single page applications
(SPA) it has become widely well known and is the top choice for many
developers for creating dynamic html pages. In order to be able to program
in AngularJS you have to know HTML, CSS and JavaScript. It is maintained by
Google and the developer-community; it is under MIT license [17]. It uses
data binding which means you can attach controllers to certain parts of the
page as well as taking advantage of the MVC (Model, View, Controller)
pattern, creating a loosely coupled design to separate the three components
of the web application so that they all are independent to one another; one of
them can be changed without impacting the others and you can swap and
change components. If an application contains more than one page it can use
Client side routing in order to dynamically switch content without refreshing
the page [18]. The Batarang plugin was built by Google in 2012 to improve
the debugging of web applications built using AngularJS. It is also used with
another three popular technologies known collectively as the MEAN Stack
(MongoDB, Express, AngularJS and NodeJS). MongoDB is cross platform
oriented database, it uses a JavaScript/JSON style syntax; it is open source.
Express is a server framework that is used for building single page web
applications and is expandable via plugins. NodeJS is cross platform runtime
environment for server side applications, it’s open source. As we can see all
the technologies used in the MEAN stack are open source suggesting the
reason for its huge community and popularity.

\subsection{Node}
\bigbreak
Is an open source cross platform runtime environment for developing server-side web applications. It has been founded and released in 2009 and interprets googles V8 JavaScript engine. Some of the corporate users from industry include IBM, LinkedIn, PayPal, SAP, Yahoo so it’s well adopted by heavy hitters in industry. Originally Node.js was only supported by linux operating systems, which enforces the reason for using Debian/linux operating system for this project.


\section{Components}
\subsection{Yeoman/Modules}
\bigbreak
Yeoman is an open source client side development stack allowing developers web applications quickly not worrying about initial process of setting up it includes all industry standards such bootstrap responsive design etc. Some of the tools used in conjunction with yeoman generator.

\begin{itemize}

	\item Grunt 
	\item Gulp
	\item Bower
	\item Npm

\end{itemize}

When yeoman generator is initialized it generates full web application with all of the main structure. 
It generates app.js file which contains all the modules injections and any new injections need to be added to this file some of the modules used in this project:

\begin{itemize}
	
	\item 'firebase' 
	\item 'ngCookies'
	\item 'ngResource'
	\item 'ngRoute'
	\item 'ngSanitize' 
	\item 'ngTouch'
	\item 'formly'
	\item 'formlyBootstrap'
	\item 'ngAnimate'
	
\end{itemize}
\bigbreak

As well it contains Route Provider that maps all the views html pages to JavaScript controllers so each view is served one controller JavaScript class, this might prove a problem as it only allows for one script per class so services can be injected to controllers to add more scripts per one view.It also maps pages html address so this can be used in future for authentication.Code below how routing ow one html page looks like.
\bigbreak
\begin{minted}{javascript}

 .when('/view', {
 templateUrl: 'views/view.html',
 controller: 'WorkBenchController',
 controllerAs: 'fireApp'
 })

\end{minted}
\bigbreak
Testing modules of Yoeman:
\bigbreak

\begin{itemize}
	
	\item jshint 
	\item travis
	\item karma
	\item jscsrc
	
\end{itemize}
\bigbreak

JSHint, a tool that helps to detect errors and potential problems in  JavaScript code.
Karma a simple tool that allows you to execute JavaScript code in multiple real browsers.
Travis lets you test your build modules.






\subsection{Angular formly}
\subsection{Ng-repeat}
\subsection{Heroku}
\subsection{Auth0}





\begin{itemize}
\item Describe each of the technologies you used at a conceptual level. Standards, Database Model (e.g. MongoDB, CouchDB), XMl, WSDL, JSON, JAXP.
\item Use references (IEEE format, e.g. [1]), Books, Papers, URLs (timestamp) – sources should be authoritative. 
\end{itemize}



\chapter{System Design}
As many pages as needed.
\begin{itemize}
\item Architecture, UML etc. An overview of the different components of the system. Diagrams etc… Screen shots etc.
\end{itemize}

\begin{table}[h]
  \centering
  \begin{tabular}{x{2cm}p{3cm}}
    \toprule \\
    Column 1 & Column 2 \\
    \midrule \\
    Rows 2.1 & Row 2.2 \\
    \bottomrule
  \end{tabular}
  \caption{A table.}
  \label{table:mytable}
\end{table}

\chapter{System Evaluation}
As many pages as needed.
\begin{itemize}
\item Prove that your software is robust. How? Testing etc. 
\item Use performance benchmarks (space and time) if algorithmic.
\item Measure the outcomes / outputs of your system / software against the objectives from the Introduction.
\item Highlight any limitations or opportuni-ties in your approach or technologies used.
\end{itemize}

\chapter{Conclusion}
About three pages.

\begin{itemize}
\item Briefly summarise your context and ob-jectives (a few lines).
\item Highlight your findings from the evalua-tion section / chapter and any opportuni-ties identified.
\end{itemize}

