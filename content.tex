\chapter*{About this project}
\paragraph{Abstract}
A brief description of what the project is, in about two-hundred and fifty words.

\paragraph{Authors}
Explain here who the authors are.



\chapter{Introduction}
It’s a booking web application build using MEAN stack (MongoDB, Express, Angular, Node).The application is made of two websites bookAroom-user and bookAroom-admin.
BookAroom-user is using Auth0 authentication system which allows for convenient log in with gmail fb as well as sign up form with password recovery. Once user is signed in he/she can book search and sort workbenches.
BookAroom-admin is using json web-tokens identification system. It lets administrative user to add, edit and deletes existing and new workbenches.

\chapter{Context}
\begin{itemize}
\item Provide a context for your project.
\item Set out the objectives of the project
\item Briefly list each chapter / section and provide a 1-2 line description of what each section contains.
\item List the resource URL (GitHub address) for the project and provide a brief list of the main elements at the URL.
\end{itemize}

\section{Filler}
Lorem ipsum dolor sit amet, consectetur adipiscing elit. Etiam mi enim, interdum ut elit lobortis, bibendum tempus diam. Etiam turpis ex, viverra tristique finibus nec, feugiat at metus. Curabitur tempus gravida interdum. Donec ac felis a lorem scelerisque elementum. Vestibulum sit amet gravida tortor, a iaculis orci. Nam a molestie augue. Curabitur malesuada odio at mattis molestie. In hac habitasse platea dictumst. Donec eu lectus eget risus hendrerit euismod nec at orci. Praesent porttitor aliquam diam, eu vestibulum nisl sollicitudin vel. Nullam sed egestas mi.

Quisque vel erat a justo volutpat auctor a nec odio. Sed rhoncus augue sit amet nisl tincidunt, vitae cursus tellus efficitur. Class aptent taciti sociosqu ad litora torquent per conubia nostra, per inceptos himenaeos. Pellentesque et auctor dui. Fusce ornare odio ipsum, et laoreet mi molestie sed. Cras at massa sit amet ipsum gravida aliquam. Nulla suscipit porta imperdiet. Fusce eros neque, bibendum sit amet consequat non, pulvinar quis ipsum.

\subsection{More filler}
Donec fermentum sapien ac rhoncus egestas. Nullam condimentum condimentum eros sit amet semper. Nam maximus condimentum ligula. Praesent faucibus in nisi vitae tempus. Sed pellentesque eleifend ante, ac malesuada nibh dapibus nec. Phasellus nisi erat, pulvinar vel sagittis sed, auctor et magna. Quisque finibus augue elit, consequat dignissim purus mollis nec. Duis ultricies euismod tortor, nec sodales libero pellentesque et. Interdum et malesuada fames ac ante ipsum primis in faucibus.

Donec id interdum felis, in semper lacus. Mauris volutpat justo at ex dignissim, sit amet viverra massa pellentesque. Suspendisse potenti. Praesent sit amet ipsum non nibh eleifend pretium. In pretium sapien quam, nec pretium leo consequat nec. Pellentesque non dui lacus. Aenean sed massa lacinia, vehicula ante et, sagittis leo. Sed nec nisl ac tellus scelerisque consequat. Ut arcu metus, eleifend rhoncus sapien sed, consequat tincidunt erat. Cras ut vulputate ipsum.

Curabitur et efficitur augue. Proin condimentum ultrices facilisis. Mauris nisi ante, ultrices sed libero eget, ultrices malesuada augue. Morbi libero magna, faucibus in nunc vitae, ultricies efficitur nisl. Donec eleifend elementum massa, sed eleifend velit aliquet gravida. In ac mattis est, quis sodales neque. Etiam finibus quis tortor eu consequat. Nullam condimentum est eget pulvinar ultricies. Suspendisse ut maximus quam, sed rhoncus urna.

\section{Filler}
Phasellus eu tellus tristique nulla porttitor convallis. Vestibulum ac est eget diam mollis consectetur. Donec egestas facilisis consectetur. Donec magna orci, dignissim vel sem quis, efficitur condimentum felis. Donec mollis leo a nulla imperdiet, in bibendum augue varius. Quisque molestie massa enim, vitae ornare lacus imperdiet non. Donec et ipsum id ante imperdiet mollis. Nullam est est, euismod sit amet cursus a, feugiat a lectus. Integer sed mauris dolor.

Mauris blandit neque tortor, consequat aliquam nisi aliquam vitae. Integer urna dolor, fermentum ut iaculis ut, semper eu lacus. Curabitur mollis at lectus at venenatis. Donec fringilla diam ac risus imperdiet suscipit. Aliquam convallis quam vitae turpis interdum, quis pharetra lacus tincidunt. Nam dictum maximus lectus, vitae faucibus ante. Morbi accumsan velit nec massa tincidunt porttitor. Nullam gravida at justo id viverra. Mauris ante nulla, eleifend vitae sem vitae, porttitor lobortis eros.

Cras tincidunt elit id nisi aliquam, id convallis ex bibendum. Sed vel odio fringilla, congue leo quis, aliquam metus. Nunc tempor vehicula lorem eu ultrices. Curabitur at libero luctus, gravida lectus sed, viverra mi. Cras ultrices aliquet elementum. Pellentesque habitant morbi tristique senectus et netus et malesuada fames ac turpis egestas. Sed metus ante, suscipit sit amet finibus ut, gravida et orci. Nunc est odio, luctus quis diam in, porta molestie magna. Interdum et malesuada fames ac ante ipsum primis in faucibus. Mauris pulvinar lacus odio, luctus tincidunt magna auctor ut. Ut fermentum nisl rhoncus, tempus nulla eget, faucibus tortor. Suspendisse eu ex nec nunc mollis pulvinar. Nunc luctus tempus tellus eleifend porta. Nulla scelerisque porttitor turpis porttitor mollis.

Duis elementum efficitur auctor. Nam nisi nulla, fermentum sed arcu vel, posuere semper dui. Fusce ac imperdiet felis. Aenean quis vestibulum nisl. Integer sit amet tristique neque, at suscipit tortor. Morbi et placerat ante, vel molestie dui. Vivamus in nibh eget massa facilisis accumsan. Nunc et purus ac urna fermentum ultrices eget sit amet justo. Class aptent taciti sociosqu ad litora torquent per conubia nostra, per inceptos himenaeos. Cras elementum dui nunc, ac tempor odio semper et. Ut est ipsum, sollicitudin eleifend nisl eu, scelerisque cursus nunc. Nam at lectus vulputate, volutpat tellus vel, pharetra mauris. Integer at aliquam massa, at iaculis sem. Morbi nec imperdiet odio. In hac habitasse platea dictumst.

Mauris a neque lobortis, venenatis erat ut, eleifend quam. Nullam tincidunt tellus quis ligula bibendum, a malesuada erat gravida. Phasellus eget tellus non risus tincidunt sagittis condimentum quis enim. Donec feugiat sapien sit amet tincidunt fringilla. Vivamus in urna accumsan, vehicula sem in, sodales mauris. Aenean odio eros, tristique non varius id, tincidunt et neque. Maecenas tempor, ipsum et sollicitudin rhoncus, nibh eros tempus dolor, vitae dictum justo massa in eros. Proin nec lorem urna. In ullamcorper vitae felis sit amet tincidunt. Maecenas consectetur iaculis est, eu finibus mi scelerisque et. Nulla id ex varius, ultrices eros nec, luctus est. Aenean ac ex eget dui pretium mattis. Ut vitae nunc lectus. Proin suscipit risus eget ligula sollicitudin vulputate et id lectus.


\chapter{Methodology}
About one to two Page
Describe the way you went about your project:
\begin{itemize}
\item Agile / incremental and iterative approach to development. Planning, meetings.
\item What about validation and testing? Junit or some other framework.
\item If team based, did you use GitHub during the development process.
\item Selection criteria for algorithms, languages, platforms and technolo-gies.
\end{itemize}



\chapter{Technology Review}
\section{Introduction}
About seven to ten pages.


Developing native applications separately for each platform is a laborious
and expensive undertaking [1]. There has been an enormous increase in
different types of mobile devices, operating systems and screen sizes; which
creates issues from an application development point of view, when
developing an application for each device. If you only take into account major
operating systems out there like Android, iOS and Windows, that’s three
separate platforms to develop for and if you want to target all of the market
it would include others like Symbian Nokia, Blackberry SIM, Samsung BADA,
WebOS and HP that’s 8 platforms to target 99 percent of the market. This
literature review analysis reviews and supports four different categories of
frameworks for mobile cross platform development that has been created in
response of heterogeneity of mobile devices by referencing papers from the
ACM and IEEE. It describes trends and issues of each type and which category
and framework is leading in mobile cross platform development.

\section{Existing and Emerging Operating Systems for Mobile Devices}

There are four dominant mobile operating systems according to the IDC
2015 Report [2]. Google’s Android (82.8%) OS is which based on Linux,
Apple’s iOS (13.9%), Microsoft’s Windows Phone (2.6%) OS and the
Blackberry (0.3%) OS. There are other platforms out there like Firefox, Tizen,
Sailfish and Ubuntu Touch OS but they only account for 0.4% market share.
While Android is the dominant operating system, it is supported on multiple
2
platforms and it is in sync with their vision of “write once, run everywhere”
[3].
There are other emerging mobile devices with their own operating systems;
for instance Mozilla launched ZTE in early 2013 [4] but its market share
hasn’t grown and is still under 0.4% market share, sharing with others. So
two of the biggest players in the market are iOS and Android [5]. They both
have communities built around them, are well matured and have been
thoroughly tested. There are many open source libraries, especially for
Android. As evidence suggests Android has the biggest market share. Over
the years Android has kept growing in market share, acquiring bigger market
share each year and predictions have been made that it will stop or slow
down; the Yankee Group forecast stated in Analysis of Cross-Platform
Development [21].

\section{Categories and Frameworks of Cross Platform Development}

Categories and Frameworks of Cross Platform Development
In todays world, application development for smart devices is an evolving
field with great economic and scientific interest [6]. Cross platform is a wide
area of development and according to Spyros Xanthopouls and Stelios
Xinnogalas’ paper [6], it has split into four main categories: Interpreted
applications, hybrid applications, generated applications and web
applications. Each category containing its own framework. The reason for
this is that the user experience for each framework creates a certain level of
native look and feel for an application against the actual operating system;
running generated applications achieves the highest native look because it’s
compiled in the OS’ native language. Web applications have the lowest native
look and feel since they are trying to target all markets using web
technologies (HTML5, CSS, JS). There are various frameworks emerging, to
name a few according to Oliver Le Goaer and Sacha Waltham’s paper [7]:
Rhodes, LiveCode, PhoneGap, Titanium, Tabris, Neomades XMLVM, Canappi,
APPlause, MoSync SDK, Codename One and Marmalade SDK. Few are more
matured then others and some well supported frameworks are Titanium and
PhoneGap. The problem that Cross-Platform is trying to solve is well
described in paper [9] and it indicates as follows:
3
“A mobile application may require to be developed several times,
one for each supported platform, thus dramatically increasing
the required time and skills for developers, and finally, the cost
of production. A solution is represented by Framework for cross
platform development.”

\section{Interpreted Applications}

These applications are interpreted natively to a platform that it’s running
on, so if it’s running on Android it will be interpreted in Java and if running
on iOS it will be Objective-C, thus creating a one hundred percent native look
and feel in the application. Other benefits include being efficient and running
smoothly; since the app is interpreted natively. One of the new frameworks
for interpreted applications is MD 2[13]. One of the more mature frameworks
would be Titanium which is discussed in more detail below.

\subsection{Titanium}

One of its most valuable features is that is is open source and has been in
development since 2006. It was released in 2008 by the Appcelerator
company and is a commercially supported product with its source code
released under the Apache 2 license [8]. It can be installed on multiple
operating systems including Windows, Macintosh and Linux. It allows the
creation of Android, iOS, Windows Phone, BlackBerry OS and Tizen apps
from a single code base. It’s based on the MVC pattern (Model View
Controller) so it’s a loosely coupled stable design. Only one JavaScript
language is now required to develop with this framework; this makes it easy
to learn and powerful, you can program all three aspects of your application:
Model, View and Controller in just one language and with the new JavaScript
ECMAScript standard 6 coming out this should make it even more robust. It
allows the use of Cloud (Server) services as a ready to use mobile backend.
When the Titanium application is compiled the engine processes the
JavaScript and builds the appropriate native application for the specific
platforms (iOS uses Objective-C, Android uses Java) thus ensuring a native
look and feel for the application. One of the downsides is that support is
limited on iOS and Android [9]. The advantage of this framework is the ability
to compile the application with the native APIs; it provides a wider set of
native device functionalities that web applications cannot provide [10].

\section{Hybrid Applications}

Hybrid applications are primarily built using HTML5, they behave like a
website. The way they work is that they have a container within the target
platform, named UIWebView in iOS and WebView in Android; this allows
them to use phone features such as the accelerometer, GPS and camera.
These new capabilities were able to be implemented with the advent of
HTML5. The user interface is not generated natively as previous seen in
interpreted applications, so technology like jQueryMobile and Sencha Touch
[12] are used to achieve native effect but they are not one hundred per cent
native. One of the more popular frameworks is PhoneGap which is based on
top of Cordova, described in more detail below.

\subsection{PhoneGap}

PhoneGap which is based on top of Apache Cordova was created by a
company Nitoby and is supported by Adobe Systems; it is open source. The
programmer needs to know HTML, CSS and JavaScript to develop
applications using this framework. PhoneGap allows deployments to many
different platforms including Android, iOS, Windows Phone, BlackBerry,
WebOS Symbian and Bada [11]. PhoneGap is a hybrid application because it’s
not pure HTML/JavaScript. PhoneGap has a bridge between target platforms
such as Android, iOS and Windows Phone which connects the JavaScript API
to the target platform device and allows developers to use such features as
the camera, accelerometer, network, storage and others. The place where
this framework is lacking is that the developer has to create his own style
sheets to make the application feel native to its target platform, there are
tools available for that such as: JQuery Mobile and Sencha Touch [12].

\subsection{Ionic}

Another hugely successful framework is the Ionic Framework. It is also
based on top of Apache Cordova. Cordova deals with the low level hardware
hooks allowing Ionic to use various hardware features on a device, such as
the accelerometer, gps and camera.
Ionic was created by Drifty Co and is an open source framework released
in 2013, it is based on AngularJS and is open source under the MIT license.
Ionic deals mostly with the visual representation of the application, you
develop the application once and you can then compile it to either iOS or
Android and it keeps the native look and feel. There are many widgets
available via Ionics library in order to achieve the same standard as native
5
apps. You would not be able to discern many Ionic apps against the native
ones. They also include services for analytics and push notifications
All you need to know in order to develop Ionic applications is CSS, HTML5,
and JavaScript. Applications are than distributed to app stores such as
Android and iOS. Ionic requires NPM (node package manager) in order to
install various plugins. Ionic seems to be one of the most popular cross
platform development frameworks for mobile at the moment.

\section{Generated Applications}

These applications are compiled natively, for that reason they achieve
high performance and generate a native user interface. One of the
frameworks used is Applause [14]. Applause is open source and is based on
Xtext but not much development has been done on the framework [13].
Applause is explained in more detail down below.

\subsection{Applause}

Applause is based on model driven software development (MDSD). It
includes tools to translate programming languages. It uses cross-compilers
(XMLVN) [15] and transforms Android applications in Java. Applause is
under EPL license which integrates with Eclipse and IDE support. There was
very little information available on Applause, there is currently only one
GitHub link [14].

\section{Web Applications}

Web applications are browser-based applications running in a browser
using HTML5. WebHooks allow developers to access the hardware on a
phone, this was unavailable before HTML5 [16], also it allows other features
such as web storage, indexed database APIs, file APIs, web SQL Databases and
Offline Web GeoLocations [16]. This makes web applications more mobile
friendly and compatible and allows them to use the full range of phone
features. They do not require installation or any upgrades as it contains a one
to many relationship (one server, many clients) so any updates are done on
the server side and all clients get updated, but the network is required at all
times in order to access the application. It lacks the native look and feel of
target platforms such as Android, iOS or Windows Phone, altough there are
many tools out there trying to solve the problem by simulating a native look
such as Xui, JQyeryMobile, Sencha Touch, JQTouch and WebApp.net. Some
6
frameworks developing web applications include AngularJS, Ruby on Rails,
Django and Drupal. AngularJS is explained in more detail down below.

\subsection{AngularJS}

It was developed by Misko Hevery in 2009 at Brath Tech LLC. It is now an
open source framework mainly used for developing single page applications
(SPA) it has become widely well known and is the top choice for many
developers for creating dynamic html pages. In order to be able to program
in AngularJS you have to know HTML, CSS and JavaScript. It is maintained by
Google and the developer-community; it is under MIT license [17]. It uses
data binding which means you can attach controllers to certain parts of the
page as well as taking advantage of the MVC (Model, View, Controller)
pattern, creating a loosely coupled design to separate the three components
of the web application so that they all are independent to one another; one of
them can be changed without impacting the others and you can swap and
change components. If an application contains more than one page it can use
Client side routing in order to dynamically switch content without refreshing
the page [18]. The Batarang plugin was built by Google in 2012 to improve
the debugging of web applications built using AngularJS. It is also used with
another three popular technologies known collectively as the MEAN Stack
(MongoDB, Express, AngularJS and NodeJS). MongoDB is cross platform
oriented database, it uses a JavaScript/JSON style syntax; it is open source.
Express is a server framework that is used for building single page web
applications and is expandable via plugins. NodeJS is cross platform runtime
environment for server side applications, it’s open source. As we can see all
the technologies used in the MEAN stack are open source suggesting the
reason for its huge community and popularity.

\section{ User Interface User Experience}

There are many different operating systems out there like Google’s
Android, Apple’s iOS, Microsoft ‘s Windows Phone OS, Nokia ‘s Symbian,
Blackberry’s OS, Samsung’s BADA, WebOS and HP. The devices these OS’ run
on all have many different screen sizes. Each platform has a different user
interaction experience and two main platforms out there would be Android
and iOS; both of them have a different user experience. It is emphasised to
programmers that they should follow each platforms various rules and
guidelines, especially with iOS [19]. The mobile user interface is very
different to that of the desktop and some companies (like Microsoft) tried to
merge the experience with the Windows Metro GUI (Graphical User
Interface) which was released in Windows 8, they quickly realised this was a
7
mistake after the public backlash and so they rolled back to the classic look
for Windows 10. Many users like the distinction between the desktop and the
mobile interface. Native applications always provide the highest level of
native user experience because it’s compiled to native code for the platform
that is being used on. As described in earlier sections there are frameworks
that fully achieve a native look for applications, a few of the examples would
be Titanium and Applause. There are others which do not fully achieve native
the native look, like PhoneGap and AngularJS but they use various libraries
to achieve more of a native look, libraries such as Xui, JQyeryMobile, Sencha
Touch, JQTouch and WebApp.net. According to results produced by paper
[20] a sufficient level of native user interface can be obtained if an
appropriate framework is picked to develop an application.

\section{ Conclusion}
9. Conclusion
“Write once run anywhere” [3] is a principal that was coined by Sun
Microsystems when they were developing java, which is exactly the same as
what all of these frameworks are trying to achieve. We still see heterogeneity
within cross platform development as it is split into four groups as described
earlier on. We see some categories gaining huge popularity and being
adopted by the community very well, like web applications with HTML5
allowing developers to use the phones hardware. With popular web
application frameworks like AngularJS maintained by Google and as part of a
bigger bundle with the MEAN Stack and with the new release of JavaScript’s
ECMAScript standard 6 it will make it more usable, robust and will add extra
capabilities. While some other frameworks such as Applause came into
existence and didn’t gain much interest or popularity and not much
development is being done as agreed by the developers from the Applause
development team referenced in following paper [13]. So what we see here
is a lot of development in Cross-Platform frameworks with some of them
aiming at big markets like Android and iOS and others trying to cover all the
operating systems. The number of different types of mobile devices keeps
increasing and with the rise of IoT (internet of things) more and more devices
are getting connected to the Internet, which leads to an exponential curve,
meaning more platforms and more cross platform development.

\begin{itemize}
\item Describe each of the technologies you used at a conceptual level. Standards, Database Model (e.g. MongoDB, CouchDB), XMl, WSDL, JSON, JAXP.
\item Use references (IEEE format, e.g. [1]), Books, Papers, URLs (timestamp) – sources should be authoritative. 
\end{itemize}

\section{XML}
Here's some nicely formatted XML:
\begin{minted}{xml}
<this>
  <looks lookswhat="good">
    Good
  </looks>
</this>
\end{minted}

\chapter{System Design}
As many pages as needed.
\begin{itemize}
\item Architecture, UML etc. An overview of the different components of the system. Diagrams etc… Screen shots etc.
\end{itemize}

\begin{table}[h]
  \centering
  \begin{tabular}{x{2cm}p{3cm}}
    \toprule \\
    Column 1 & Column 2 \\
    \midrule \\
    Rows 2.1 & Row 2.2 \\
    \bottomrule
  \end{tabular}
  \caption{A table.}
  \label{table:mytable}
\end{table}

\chapter{System Evaluation}
As many pages as needed.
\begin{itemize}
\item Prove that your software is robust. How? Testing etc. 
\item Use performance benchmarks (space and time) if algorithmic.
\item Measure the outcomes / outputs of your system / software against the objectives from the Introduction.
\item Highlight any limitations or opportuni-ties in your approach or technologies used.
\end{itemize}

\chapter{Conclusion}
About three pages.

\begin{itemize}
\item Briefly summarise your context and ob-jectives (a few lines).
\item Highlight your findings from the evalua-tion section / chapter and any opportuni-ties identified.
\end{itemize}

